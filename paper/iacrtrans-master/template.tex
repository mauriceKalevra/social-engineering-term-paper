%%%% 1. DOCUMENTCLASS %%%%
\documentclass[journal=tosc,final]{iacrtrans}
%%%% NOTES:
% - Change "journal=tosc" to "journal=tches" if needed
% - Change "submission" to "final" for final version
% - Add "spthm" for LNCS-like theorems


%%%% 2. PACKAGES %%%tsdss
\usepackage[left, pagewise,edtable]{lineno}
\usepackage{blt}
\usepackage{graphicx}
\usepackage{framed} 
\usepackage{xcolor}
\usepackage{tcolorbox}
\usepackage{xcolor} 
\colorlet{shadecolor}{gray!25}
\definecolor{mshadecolor}{rgb}{0.7421875,0.7421875,0.7421875}
\setlength{\OuterFrameSep}{10pt}
%%%% 3. AUTHOR, INSTITUTE %%%
\author{Moritz Rupp}
\institute{
  Hochschule Albstadt-Sigmaringen, Albstadt, Germany, \email{ruppmori@hs-albsig.de}
  
}



%%%% 4. TITLE %%%%
\title{Technische Lösungsansätze gegen Phishing}

\author{Moritz Rupp}

\begin{document}

\maketitle
\author


%%% 5. KEYWORDS %%%%s
\keywords{Social-Engineering \and Phishing \and IT-Security \and Container \and DNS }


%%%% 6. ABSTRACT %%%%s
\begin{abstract} Phishing zählt nach wie vor zu den häufigsten Methoden bei Cyberangriffen. Hierbei werden versucht vertrauliche Informationen wie Passwörter oder Kreditkartennummern abzugreifen, indem sich Angreifer als vertrauenswürdige Quelle ausgeben. Da der Mensch das Hauptziel an diesem Angriff ist, wird hauptsächlich durch Schulungen versucht, dem entgegen zu halten. in der Arbeit wird jedoch die Möglichkeit untersucht durch konkret teechnische Maßnahmen abhilfe zu schaffen. \dots   \end{abstract}

%%%% 7. PAPER CONTENT %%%%
\section{Einführung}
Phishing ist\dots\\
Angriffslanndschaft, Phishing techniken\dots\\
Taxonomy of defense against phishing attacks\\
Stand der Forschung etc\dots

\newpage
\bibliographystyle{alpha}
\bibliography{ref.bib}
\end{document}
